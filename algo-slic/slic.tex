\documentclass[12pt,a4paper]{article}

\usepackage{amsmath}
\usepackage[font=footnotesize]{caption}
\usepackage[section]{algorithm}
\captionsetup[algorithm]{font=footnotesize}
\usepackage[numbered]{algo}

\begin{document}

	\begin{algorithm}[t!]
		\begin{algo}{SLIC}{\label{algo:superpixel-segmentation-slic}\qinput{color image $I$, number of superpixels $K$}\qoutput{superpixel segmentation $S$}}
			initialize superpixel centers on a regular grid with step size $R$\\
			move centers to low-gradient magnitude positions\\
			\qrepeat\\
				\qfor $k = 1$ \qto $K$\\
					\qforeach pixel $x_n$ in a $2R \times 2R$ neighborhood around $\mu(S_k)$\\
						\qif $x_n$ is unassigned\\
							\qthen $S_k$ \qlet $S_k \cup \{x_n\}$\qfi\\
						\qelseif $d(x_n, S_k) < d(x_n, S_{s(x_n)})$\\
							\qthen $S_k$ \qlet $S_k \cup \{x_n\}$ and $S_{s(x_n)}$ \qlet $S_{s(x_n)} - \{x_n\}$\qfi\qrof\qrof
			\quntil nothing changes \qcom{or maximum number of iterations reached.}\\
			enforce connectivity\\
			\qreturn $S$
		\end{algo}
		\caption[The basic algorithm of \textbf{SLIC} \cite{AchantaShajiSmithLucchiFuaSuesstrunk}.]{\textbf{SLIC} is implemented as local $K$-means clustering. Here, local means that for each superpixel only pixels in a $2R \times 2R$ window around the superpixel's center are of interest. After clustering, \textbf{SLIC} needs to enforce connectivity.}
		\label{fig:superpixel-segmentation-slic-algorithm}
	\end{algorithm}

	\begin{thebibliography}{1}
		\bibitem{AchantaShajiSmithLucchiFuaSuesstrunk}
		R. Achanta, A. Shaji, K. Smith, A. Lucchi, P. Fua, S. Süsstrunk.
		\emph{SLIC superpixels}.
		Technical report, École Polytechnique Fédérale de Lausanne, Lusanne, Switzerland, June 2010.
	\end{thebibliography}

\end{document}