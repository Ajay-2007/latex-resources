\documentclass[12pt,a4paper]{article}

\usepackage{amsmath}
\usepackage[font=footnotesize]{caption}
\usepackage[section]{algorithm}
\captionsetup[algorithm]{font=footnotesize}
\usepackage[numbered]{algo}

\begin{document}

	\begin{algorithm}[t]
	\begin{algo}{VCCS}{\label{algo:superpixel-segmentation-depth-vccs}\qinput{voxelized point cloud, supervoxel resolution $R$, search radius $R_s$}\qoutput{superpixel segmentation $S$}}
		place supervoxel centers on a regular grid with step size $R$\\
		discard unnecessary supervoxel centers based on the search radius $R_s$\\		
		move supervoxel centers to low gradient magnitude positions\\
		%\qcom{Eventually, move supervoxel centers to mean position within local neighborhood}\\
		\qfor $t = 1$ \qto $T$ \qcom{$T$ is the maximum depth.}\\
			\qfor $k = 1$ \qto $K$\\
				perform one step breadth first search for supervoxel $S_k$\\
				update supervoxel center\qrof\qrof\\
		derive superpixel segmentation $S$ by backprojecting the supervoxels\\
		\qreturn $S$
	\end{algo}
	\caption[The supervoxel algorithm \textbf{VCCS} \cite{PaponAbramovSchoelerWoergoetter}.]{\textbf{VCCS} uses $K$-means clustering based on breadth-first search beginning at the supervoxel centers to assign each voxel to a supervoxel. This way, \textbf{VCCS} ensures that the supervoxels represent connected components within the $26$-adjacency graph derived from the voxelized point cloud. The below algorithm can easily be adapted to return a supervoxel segmentation instead of a superpixel segmentation.}
	\label{fig:superpixel-segmentation-depth-vccs-breadth-first}
\end{algorithm}

	\begin{thebibliography}{1}
		\bibitem{PaponAbramovSchoelerWoergoetter}
		J. Papon, A. Abramov, M. Schoeler, F. Wörgötter.
		\emph{Voxel cloud connectivity segmentation - supervoxels for point clouds}.
		Conference on Computer Vision and Pattern Recognition, 2013.
	\end{thebibliography}

\end{document}