\documentclass[12pt,a4paper]{article}

\usepackage[utf8]{inputenc}
\usepackage{amsmath}
\usepackage[font=footnotesize]{caption}
\usepackage[section]{algorithm}
\captionsetup[algorithm]{font=footnotesize}
\usepackage[numbered]{../packages/algo}

% Following packages only required for this sample, not for using algo.sty in the first place!
\usepackage{color}
\usepackage{listings}
\lstset{language=TeX}

\definecolor{dkgreen}{rgb}{0,0.6,0}
\definecolor{dkgray}{rgb}{0.25,0.25,0.25}

\lstset{%
  	backgroundcolor=\color{white},
  	basicstyle=\footnotesize,
  	breakatwhitespace=false,
  	breaklines=true,
  	captionpos=b,
  	commentstyle=\color{dkgreen},
  	deletekeywords={...},
  	escapeinside={\%*}{*)},
  	extendedchars=true,
  	frame=single,
  	keepspaces=true,
  	keywordstyle=\color{blue},
  	language=Octave,
  	morekeywords={*,...},
  	numbers=left,
  	numbersep=5pt,
  	numberstyle=\tiny\color{dkgray},
  	rulecolor=\color{dkgray},
  	showspaces=false,
  	showstringspaces=false, 
  	showtabs=false,
  	stepnumber=1,
  	tabsize=2,
  	title=\lstname
}

\begin{document}

	\section{Quick Shift}

	To use the \lstinline!alog.sty! package for a book, update the following lines of \lstinline!algo.sty!:
	
\begin{lstlisting}
% Set counter to include chapter:
% \renewcommand{\thealgorithm}{\thechapter .\arabic{algo}}
\renewcommand{\thealgorithm}{\thesection .\arabic{algo}}
\end{lstlisting}

	\begin{algorithm}[h]
		\begin{algo}{QS}{\label{algo:related-work-qs}\qinput{color image $I$}\qoutput{superpixel segmentation $S$}}
			\qfor $n = 1$ \qto $N$\\
				initialize $t(x_n) = \boldsymbol 0$\qrof\\
			\qfor $n = 1$ \qto $N$\\
				\qcom{$N_R(x_n)$ is the set of all pixels in the local neighborhood of size $R \times R$ around pixel $x_n$:}\\
				calculate $p(x_n) = \sum_{x_m \in N_R(x_n)} \exp\left(\frac{-d(x_n,x_m)^2}{(2/3) R}\right)$ \qrof\\
			\qfor $n = 1$ \qto $N$\\
				set $t(x_n) = \arg\max_{x_m \in N_R(x_n): p(x_m) > p(x_n)} \{p(x_m)\}$\qrof\\
				\qcom{$t$ maps each pixel to its neighbor $x_m$ with highest $p(x_m)$ if $p(x_m) > p(x_n)$;}\\
			\qcom{$t$ can be interpreted as forest, where all pixels $x_n$ with $t(x_n) = \boldsymbol 0$ are roots.}\\
			derive superpixel segmentation $S$ from $t$\\
			\qreturn $S$
		\end{algo}
		\caption{The superpixel algorithm \textbf{QS} proposed in \cite{QuickShift}.}
		\label{fig:related-work-qs-algorithm}
	\end{algorithm}

	\begin{thebibliography}{1}
		\bibitem{QuickShift}
		A. Vedaldi,
		S. Soatto,
		\emph{Quick shift and kernel methods for mode seeking},
		ECCV,
		2008.
	\end{thebibliography}

\end{document}