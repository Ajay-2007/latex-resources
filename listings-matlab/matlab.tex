\documentclass[twoside,11pt,a4paper]{article}

\usepackage{amsmath, amssymb, latexsym}

\usepackage{xcolor}
\definecolor{dkgreen}{rgb}{0,0.6,0}
\definecolor{gray}{rgb}{0.5,0.5,0.5}
\definecolor{mauve}{rgb}{0.58,0,0.82}

\usepackage{listings}
\lstset{%
numbers=left,
frame=none,
basicstyle=\footnotesize\ttfamily,
numbers=left,
numberstyle=\tiny,
numbersep=15pt,tabsize=4,
flexiblecolumns=true,
keywordstyle=\color{blue},
commentstyle=\color{dkgreen}, 
stringstyle=\color{mauve},
numberstyle=\tiny\color{gray},
language=Matlab,
breaklines=true,
breakatwhitespace=true,
morekeywords={*,num,String,var,library,get,set} ,
}

\begin{document}
\begin{lstlisting}[label={lst:train-two-layer-perceptron}]
function [hiddenWeights, outputWeights, error] = trainStochasticSquaredErrorTwoLayerPerceptron(activationFunction, dActivationFunction, numberOfHiddenUnits, inputValues, targetValues, epochs, batchSize, learningRate)
% trainStochasticSquaredErrorTwoLayerPerceptron Creates a two-layer perceptron
% and trains it on the MNIST dataset.
%
% INPUT:
% activationFunction             : Activation function used in both layers.
% dActivationFunction            : Derivative of the activation
% function used in both layers.
% numberOfHiddenUnits            : Number of hidden units.
% inputValues                    : Input values for training (784 x 60000)
% targetValues                   : Target values for training (1 x 60000)
% epochs                         : Number of epochs to train.
% batchSize                      : Plot error after batchSize images.
% learningRate                   : Learning rate to apply.
%
% OUTPUT:
% hiddenWeights                  : Weights of the hidden layer.
% outputWeights                  : Weights of the output layer.
% 

    % The number of training vectors.
    trainingSetSize = size(inputValues, 2);
    
    % Input vector has 784 dimensions.
    inputDimensions = size(inputValues, 1);
    % We have to distinguish 10 digits.
    outputDimensions = size(targetValues, 1);
    
    % Initialize the weights for the hidden layer and the output layer.
    hiddenWeights = rand(numberOfHiddenUnits, inputDimensions);
    outputWeights = rand(outputDimensions, numberOfHiddenUnits);
    
    hiddenWeights = hiddenWeights./size(hiddenWeights, 2);
    outputWeights = outputWeights./size(outputWeights, 2);
    
    n = zeros(batchSize);
    
    figure; hold on;

    for t = 1: epochs
        for k = 1: batchSize
            % Select which input vector to train on.
            n(k) = floor(rand(1)*trainingSetSize + 1);
            
            % Propagate the input vector through the network.
            inputVector = inputValues(:, n(k));
            hiddenActualInput = hiddenWeights*inputVector;
            hiddenOutputVector = activationFunction(hiddenActualInput);
            outputActualInput = outputWeights*hiddenOutputVector;
            outputVector = activationFunction(outputActualInput);
            
            targetVector = targetValues(:, n(k));
            
            % Backpropagate the errors.
            outputDelta = dActivationFunction(outputActualInput).*(outputVector - targetVector);
            hiddenDelta = dActivationFunction(hiddenActualInput).*(outputWeights'*outputDelta);
            
            outputWeights = outputWeights - learningRate.*outputDelta*hiddenOutputVector';
            hiddenWeights = hiddenWeights - learningRate.*hiddenDelta*inputVector';
        end;
        
        % Calculate the error for plotting.
        error = 0;
        for k = 1: batchSize
            inputVector = inputValues(:, n(k));
            targetVector = targetValues(:, n(k));
            
            error = error + norm(activationFunction(outputWeights*activationFunction(hiddenWeights*inputVector)) - targetVector, 2);
        end;
        error = error/batchSize;
        
        plot(t, error,'*');
    end;
end
\end{lstlisting}
\end{document}