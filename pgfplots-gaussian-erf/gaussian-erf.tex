\documentclass[a4paper,12pt]{article}

\usepackage{subcaption}
\usepackage{pgfplots}
\usepackage{tikz}
\usepgfplotslibrary{fillbetween}

% https://tex.stackexchange.com/questions/11368/bell-curve-gaussian-function-normal-distribution-in-tikz-pgf
\makeatletter
\pgfmathdeclarefunction{gauss}{2}{%
  \pgfmathparse{1/(#2*sqrt(2*pi))*exp(-((x-#1)^2)/(2*#2^2))}%
}
\makeatother

% https://tex.stackexchange.com/questions/107423/erf-function-in-latex
\makeatletter
\pgfmathdeclarefunction{erf}{1}{%
  \begingroup
    \pgfmathparse{#1 > 0 ? 1 : -1}%
    \edef\sign{\pgfmathresult}%
    \pgfmathparse{abs(#1)}%
    \edef\x{\pgfmathresult}%
    \pgfmathparse{1/(1+0.3275911*\x)}%
    \edef\t{\pgfmathresult}%
    \pgfmathparse{%
      1 - (((((1.061405429*\t -1.453152027)*\t) + 1.421413741)*\t 
      -0.284496736)*\t + 0.254829592)*\t*exp(-(\x*\x))}%
    \edef\y{\pgfmathresult}%
    \pgfmathparse{(\sign)*\y}%
    \pgfmath@smuggleone\pgfmathresult%
  \endgroup
}
\makeatother

\begin{document}
  
\begin{figure}
  \centering
  \begin{subfigure}[t]{0.48\textwidth}
    \centering
    \begin{tikzpicture}
      \begin{axis}[
          every axis plot post/.append style={
          mark=none,domain=-3:2,samples=50,smooth},
          axis x line*=bottom,
          axis y line*=left,
          enlargelimits=upper,
          ymax=1.1, ymin=0,
          ylabel=$p(y)$,
          xlabel=$y$,
          legend style={at={(0.65,1)},anchor=north west},
          height=5cm,
          width=7cm,
        ]
      
        \addplot[name path=g,blue] {gauss(-0.75,0.5)};
        \addlegendentry{$p(y)$};
        %\addplot[red] {gauss(1,0.5)};
        %\addlegendentry{$\mu_j(z) = -0.5; \sigma^2 = 0.5$};
        \addplot[blue,dashed,mark=none,forget plot] coordinates {(-0.75, 0) (-0.75, 1)};
        \node at (axis cs:-0.75,1.1) {$\mu_i(z)$};
      
        \addplot[black,dashed,mark=none,forget plot] coordinates {(0, 0) (0, 1)};
        \node at (axis cs:0,1.1) {$0$};
      
        %\addplot[red,dashed,mark=none] coordinates {(1, 0) (1, 0.9)};
        %\node at (axis cs:1,1) {$\mu_j(z)$};
      
        \path[name path=axis] (axis cs:-3,0) -- (axis cs:0,0);
        \addplot [
          thick,
          color=blue,
          fill=blue, 
          fill opacity=0.1,
          area legend, % https://tex.stackexchange.com/questions/246557/custom-area-legend-image-not-working
        ]
        fill between[
          of=g and axis,
          soft clip={domain=-3:0},
        ];
        \addlegendentry{$p(y \leq 0)$}
      \end{axis}
    \end{tikzpicture}
  \end{subfigure}\hfill
  \begin{subfigure}[t]{0.48\textwidth}
    \begin{tikzpicture}
      \begin{axis}[
          every axis plot post/.append style={
          mark=none,domain=-3:2,samples=50,smooth},
          axis x line*=bottom,
          axis y line*=left,
          enlargelimits=upper,
          ymax=1.1, ymin=0,
          ylabel=$p(y' \leq y)$,
          xlabel=$y$,
          legend style={at={(0.65,1)},anchor=north west},
          height=5cm,
          width=7cm,
        ]
      
        % 1.128379167 = 1/(0.5*sqrt(pi))
        \addplot[name path=g,blue] (\x,{0.5*(1 + erf((\x + 0.5)*1.128379167))});
        \addlegendentry{$p(y' \leq y)$};
        \addplot[black,dashed,mark=none,forget plot] coordinates {(0, -1) (0, 1)};
        \node at (axis cs:0,1.1) {$0$};
      \end{axis}
    \end{tikzpicture}
  \end{subfigure}
\end{figure}

\end{document}